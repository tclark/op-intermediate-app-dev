% Beamer slide template prepared by Tom Clark <tom.clark@op.ac.nz>
% Otago Polytechnic
% Dec 2012

\documentclass[10pt]{beamer}
\usetheme{Dunedin}
\usepackage{graphicx}
\usepackage{fancyvrb}

\newcommand\codeHighlight[1]{\textcolor[rgb]{1,0,0}{\textbf{#1}}}

\title{Introduction}

\author[IN608]{Intermediate Application Development}
\institute[Otago Polytechnic]{
  Otago Polytechnic \\
  Dunedin, New Zealand \\
  Kaiako: Tom Clark
}
\date{}
\begin{document}

%----------- titlepage ----------------------------------------------%
\begin{frame}[plain]
  \titlepage
\end{frame}

%----------- slide --------------------------------------------------%
\begin{frame}
  \frametitle{Administration}
      
   \begin{itemize}
     \item Communication will take place on Teams and email
     \item We will use Git and Github extensively. You'll need to have a GitHub account and 
     to set up Git on your computer.
     \item Course materials
       \begin{itemize}
         \item Course notes and worksheets: https://github.com/tclark/op-intermediate-app-dev
         \item Practical submissions: \url{https://classroom.github.com/a/ZOav9L3E}
         \item Project one submissions:  \url{https://classroom.github.com/a/rfDZD0UC}
         \item Project two submissions: \url{https://classroom.github.com/a/aFVjykKU}
       \end{itemize}
   \end{itemize}
      
\end{frame}



%----------- slide --------------------------------------------------%
\begin{frame}
  \frametitle{Python}

   We're going to do our programming in Python, so you'll need to 
   have Python installed on your computer
   
   \begin{itemize}
     \item Use version $\geq$ 3.6
     \item Beware of Python 2
     \item Windows: Anaconda
     \item Use Homebrew or MacPorts to install Python3
     \item Linux: Install Python3 from you package manager.
   \end{itemize}
\end{frame}

%----------- slide --------------------------------------------------%
\begin{frame}
  \frametitle{Python}

   There are three ways we'll use Python.
   
   \begin{itemize}
     \item Interactive shell
     \item Invoke the interpreter
     \item JupyterLab
     \item While we're here: pip and Pipenv
   \end{itemize}
\end{frame}

%----------- slide --------------------------------------------------%
\begin{frame}
  \frametitle{The Python Language}
  
  \begin{itemize}
    \item Developed 30 years ago by Guido van Rossum
    \item Interpreted, dynamically typed
    \item Core philosophy
      \begin{itemize}
        \item Beautiful is better than ugly
        \item Explicit is better than implicit
        \item Simple is better than complex
        \item Complex is better than complicated
        \item Redability counts
      \end{itemize}   
  \end{itemize}

\end{frame}
%----------- slide --------------------------------------------------%
\begin{frame}
  \frametitle{Python Style}

   Python has clear style guidelines.
   
   \begin{itemize}
     \item PEP 8 \url{https://www.python.org/dev/peps/pep-0008/}
     \item Key points:
       \begin{itemize}
         \item Code blocks are designated by indentation. Indents should be 4 spaces.
         \item Class names are \texttt{CamelCased}
         \item Variable names are \texttt{lower}, or \texttt{snake\_cased}.
       \end{itemize}
     \end{itemize}  
\end{frame}

%----------- slide --------------------------------------------------%
\begin{frame}
  \frametitle{Programming Activity}
  
  \begin{enumerate}
    \item Install Python, Git if necessary
    \item Clone the course materials repo.
    \item Click the GitHub classroom link for the practicals to set up your repo.
    \item Clone/initialise that repo on your machine.
    \item Add a \texttt{README} and a \texttt{.gitignore} (Python) to your repo.
    \item Add, commit, and push to your repo.
    \item Create a new branch, \texttt{01-practical} in your repo.
    \item Add a subdirectory,  \texttt{01-practical} and copy \texttt{01-practical.ipynb} from the class materials into it.
    \item Open a shell, cd to this directory, and run \texttt{jupyter lab} to open the notebook. Complete the first question.
    \item Add, commit, and push your new files.
    \item On the GitHub page for your repo, create a pull request for this commit. Identify \emph{tclark} as the reviewer.
  \end{enumerate}      
\end{frame}


%----------- slide --------------------------------------------------%
\begin{frame}[fragile]
  \frametitle{OOP Review: Access}
  
   Access modifiers - Public
   
   Class members are public by default.
   
   \begin{verbatim}
    class Cat:
        def __init__(self, name, breed):
            self.name = name
            self.breed = breed
            
        def speak(self):
              return f'Meow, my name is {self.name}'
              
     if __name__ == '__main__':
         persian = Cat('Tom', 'persian')
         persian.name = 'Jerry'
         print(persian.speak())
    \end{verbatim}     
     
\end{frame}

\begin{frame}[fragile]
  \frametitle{OOP Review: Access}
  
   Access modifiers - Private/Protected
   
   ``Private'' members can be identified with a leading underscore.
   
   \begin{verbatim}
    class Cat:
        def __init__(self, name, breed):
            self._name = name
            self._breed = breed
            
        def speak(self):
              return f'Meow, my name is {self._name}'
    \end{verbatim}
    
    We can do this with fields and methods. The interpreter doesn't enforce any access
    restrictions. The notation merely tells programmers to treat the members as private.
     
\end{frame}

\begin{frame}[fragile]
  \frametitle{OOP Review: Encapsulation}
     
   We can easily add setters and getters with the \texttt{@property} decorator.
   
   \begin{verbatim}
    class Cat:
        def __init__(self, name, breed):
            self._name = name
            self._breed = breed
        
        @property
        def name(self):
            return self._name
            
        @name.setter
        def name(self, name):
            self._name = name
            
        if __name__ == '__main__':
            persian = Cat('Tom', 'persian')
            persian.name = 'Jerry'
            print(persian.name)                
            
    \end{verbatim} 
\end{frame}

\begin{frame}[fragile]
  \frametitle{OOP Review: Inheritance}
  It is possible to extend a base class with a child class.
  
  \begin{verbatim}
   class Employee:
       def __init__(self, first_name, last_name, salary):
           self.first_name = first_name
           self.last_name = last_name
           self.salary = salary
           
       def __str__(self):
           return f'{self.first_name} {self.last_name}'
       
   class SoftwareDeveloper(Employee):
       def __init__(self, first_name, last_name, 
                    salary, prog_lang):
           super().__init__(first_name, last_name, salary)
           self.prog_lang = prog_lang  
   \end{verbatim}         
\end{frame}

\begin{frame}[fragile]
  \frametitle{OOP Review: Polymorphism}
  
  \begin{verbatim}
   class Country:
       def capital(self):
           raise NotImplementedError
           
   class NewZealand(Country):
       def capital(self):
           return 'Wellington is the capital of New Zealand.'
      
   class Brazil(Country):
       def capital(self):
           return 'Brasilia is the capital of Brazil.'
           
   class Canada(Country):
       pass
       
   \end{verbatim}         
\end{frame}

\begin{frame}[fragile]
  \frametitle{OOP Review: Polymorphism}
  
  \begin{verbatim}
    nzl = NewZealand()
    bra = Brazil()
    can = Canada()
    for country in (nzl, bra, can):
        print(country.capital())
  \end{verbatim}         
\end{frame}

\begin{frame}[fragile]
  \frametitle{OOP Review: Polymorphism}
  
  Duck Typing
  
  \begin{verbatim}
   class NewZealand:
       def capital(self):
           return 'Wellington is the capital of New Zealand.'
     
   class Brazil:
       def capital(self):
           return 'Brasilia is the capital of Brazil.'
           
   class Canada:
      pass
      
   nzl = NewZealand()
   bra = Brazil()
   can = Canada()
   for country in (nzl, bra, can):
       print(country.capital())
  \end{verbatim}         
\end{frame}

\end{document}
