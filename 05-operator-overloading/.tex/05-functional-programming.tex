% Beamer slide template prepared by Tom Clark <tom.clark@op.ac.nz>
% Otago Polytechnic
% Dec 2012

\documentclass[10pt]{beamer}
\usetheme{Dunedin}
\usepackage{graphicx}
\usepackage{fancyvrb}

\newcommand\codeHighlight[1]{\textcolor[rgb]{1,0,0}{\textbf{#1}}}

\title{Functional Programming}

\author[IN608]{Intermediate Application Development}
\institute[Otago Polytechnic]{
  Otago Polytechnic \\
  Dunedin, New Zealand \\
  Kaiako: Tom Clark
}
\date{}
\begin{document}

%----------- titlepage ----------------------------------------------%
\begin{frame}[plain]
  \titlepage
\end{frame}

%----------- slide --------------------------------------------------%
\begin{frame}
  \frametitle{Introduction}
  
  Most of us are primarily familiar with object-oriented programming. It's 
  principal abstractions are around classes and objects. Objects are frequently used
  to maintain state. Python is an object-oriented language.
  
  \vspace{5mm}
  There are completely different types of programming and programming languages. One
  of them that is used today is \emph{functional programming}. Examples of functional
  programming languages include
  \begin{enumerate}
    \item Lisp
    \item Haskell
    \item Javascript
  \end{enumerate}
  
  \vspace{5mm}
  Python is not a functional language, but it does support some functional programming
  features.  
\end{frame}

%----------- slide --------------------------------------------------%
\begin{frame}[fragile]
  \frametitle{Key idea: Pure Functions}
  
  A \emph{pure function} has two properties.
  \begin{enumerate}
    \item It always returns the same result when given the same arguments;
    \item It has no side effects.
  \end{enumerate}
  
  \vspace{5mm}
  Here is an example of a pure function.
  \vspace{5mm}
  \begin{verbatim}
    def square(x):
        return x * x
  \end{verbatim}
  
  \vspace{5mm}
  Functional programming relies on pure functions.
        
\end{frame}

%----------- slide --------------------------------------------------%
\begin{frame}[fragile]
  \frametitle{Sidebar: Impure Functions}
  
  This function is not pure:
  
  \vspace{5mm}
  \texttt{random.randint(0, 100)}
  
  \vspace{5mm}
  Given the same arguments (0, 100) it will not always return the same 
  result.
  
  \vspace{10mm}
  Neither is this one
  \begin{verbatim}
    l = [1, 2, 3]
    l.pop()
  \end{verbatim}
  \vspace{5mm}
  Besides returning the last item on the list, it also \textbf{changes} the list by removing 
  the last item. This is a side effect.  
\end{frame}

%----------- slide --------------------------------------------------%
\begin{frame}
  \frametitle{Key Idea: First Class Functions}
  
  Functions in Python are first-class values.
  \begin{itemize}
    \item We can assign a function's value to a variable.
    \item Functions can be passed as arguments to other functions.
    \item Functions can be returned from other functions.
  \end{itemize}  
  
  \vspace{5mm}
  First class functions are not required for functional programming, but
  they make it easier.
  
  \end{frame}

%----------- slide --------------------------------------------------%
\begin{frame}[fragile]
  \frametitle{First Class Functions: Examples}
  
  \begin{verbatim}

  def double(x):
      return x * 2
      
  times_two = double # <- notice the lack of brackets
  print(times_two(2)) # 4    
  
  def apply_twice(fn, arg):
      return fn(arg), fn(arg)
  
  apply_twice(double, 2)  # returns 4, 4
  
  def make_multiplier(factor):
      def mult(x):
          return x * factor
      return mult
      
  times_two = make_multiplier(2)         
    
  \end{verbatim}
\end{frame}

%----------- slide --------------------------------------------------%
\begin{frame}[fragile]
  \frametitle{Sidebar: Closures}
  
  Hey, wait...
  \begin{verbatim}

    def make_multiplier(factor):
      def mult(x):
          return x * factor
      return mult
      
   \end{verbatim}
   Notice how our function \texttt{mult()} uses the local variable
   \texttt{factor} that goes out of scope when the function \texttt{make\_multiplier()}
   exits? 
   
   The function encloses its lexical scope at the time it is defined. We call this
   a \emph{closure}. 
   
\end{frame}

%----------- slide --------------------------------------------------%
\begin{frame}[fragile]
  \frametitle{One More Idea: Lambdas}
  
  Sometimes we need a small function in a very specific context.
  
  The following are equivalent
  \begin{verbatim}
  def double(x):
      return x * 2
      
  double = lambda x: x * 2   # <- Note the lack of "return". 
  \end{verbatim}
  
  Typically we use lambdas when we need a function as an argument to another 
  function
  
  \begin{verbatim}
  
  foo(lambda x: x * 6, [2, 7, 'cat'])
  
  \end{verbatim}
  Another name for lambda is \emph{anonymous function}.
  
  N.B.: In Python lambdas are limited to only one expression.
  
\end{frame}
%----------- slide --------------------------------------------------%
\begin{frame}
  \frametitle{Programming Activity}
  
  \begin{enumerate}
    \item Pull the course materials repo.
    \item Create a new branch, \texttt{05-practical} in your practicals repo.
    \item Add a subdirectory,  \texttt{05-practical} and copy \texttt{04-practical.ipynb} from the class materials into it.
    \item Open a shell, cd to this directory, and run \texttt{jupyter notebook} to open the notebook. Complete the first questions.
    \item We will discuss results in 30ish minutes.
  \end{enumerate}      
\end{frame}



%----------- slide --------------------------------------------------%
\begin{frame}[fragile]
  \frametitle{Map}
  
  \texttt{map(function, iterable) -> iterator}
  
  Returns iterator which applies \texttt{function} to elements of \texttt{iterable}, yielding the results
    
  \begin{verbatim}
  nums = [1, 2, 3, 4, 5]
  cubes = map(lambda x: x ** 3, nums)
  print(cubes)  # [1, 8, 27, 64, 125]
               
  \end{verbatim}           
  
\end{frame}

%----------- slide --------------------------------------------------%
\begin{frame}[fragile]
  \frametitle{Filter}
  \texttt{filter(function, iterable) -> iterator}
  
  Returns iterator which yields elements of \texttt{iterable} for which \texttt{function}
  returns \texttt{True}.

    
  \begin{verbatim}
  def is_even(num):
      returns x % 2 == 1
      
  nums = [1, 2, 3, 4, 5]
  odds = filter(is_even, nums)
  print(odds) # [1, 3, 5]            
  \end{verbatim}           
  
\end{frame}

%----------- slide --------------------------------------------------%
\begin{frame}[fragile]
  \frametitle{Reduce}
  \texttt{reduce(function, iterable) -> value}
  
  Applies \texttt{function} of two arguments cumulatively to elements of 
  iterable from left to right, reducing the iterable to a single value
  
  From the \texttt{functools} module    
  \begin{verbatim}
  from functools import reduce
            
  def add(x, y):
      return x + y
      
  nums = [1, 2, 3, 4, 5]
  sum_nums = reduce(add, nums)
  print(sum_nums) # 15            
  \end{verbatim}           
  
\end{frame}

%----------- slide --------------------------------------------------%
\begin{frame}[fragile]
  \frametitle{Partial}
  \texttt{partial(function, *args) -> partial object}
  
  Returns a function-like object (basically a function). This new function
  behaves like the original function with \texttt{*args} supplied to it. 
  
  From the \texttt{functools} module    
  \begin{verbatim}
  from functools import partial
            
  def add(x, y):
      return x + y
  
  add_two = partial(add, 2) 
  # add_two is like add(2, x)
  add_two(3)  # returns 5    
        
  \end{verbatim}           
  
\end{frame}

%----------- slide --------------------------------------------------%
\begin{frame}[fragile]
  \frametitle{List (and other) Comprehensions}
  
  
  The functionality of \texttt{map} and \texttt{filter} is combined in 
  a \emph{list comprehension}. This is why it's rare to see \texttt{map} and
  \texttt{filter} in Python. 
  
   
  \begin{verbatim}
  string = '123 Hi 456'
  nums = [int(s) for s in string if s.isdigit()]
  print(nums)  # [1, 2, 3, 4, 5, 6]        
  \end{verbatim} 
  
  This is equivalent to
  \begin{verbatim}
  string = '123 Hi 456'
  nums = []
  for s in string:
      if s.isdigit():
          nums.append(int(s))   
  \end{verbatim} 
 There are also set and dictionary comprehensions that work in a similar way.           
  
\end{frame}

\end{document}
