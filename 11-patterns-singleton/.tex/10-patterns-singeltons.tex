% Beamer slide template prepared by Tom Clark <tom.clark@op.ac.nz>
% Otago Polytechnic
% Dec 2012

\documentclass[10pt]{beamer}
\usetheme{Dunedin}
\usepackage{graphicx}
\usepackage{fancyvrb}

\newcommand\codeHighlight[1]{\textcolor[rgb]{1,0,0}{\textbf{#1}}}

\title{Introduction to Design Patterns \\ \& Singletons}

\author[IN608]{Intermediate Application Development}
\institute[Otago Polytechnic]{
  Otago Polytechnic \\
  Dunedin, New Zealand \\
  Kaiako: Tom Clark
}
\date{}
\begin{document}

%----------- titlepage ----------------------------------------------%
\begin{frame}[plain]
  \titlepage
\end{frame}

%----------- slide --------------------------------------------------%
\begin{frame}
  \frametitle{Introduction}
  
    Last time we talked about the development of a body of knowledge 
    around Object Oriented Design that included SOLID Principles.
    
    \vspace{5mm}
    Another major part of that project is the idea of \emph{Design Patterns}.
    The principal reference on that topic is the 1995 book \emph{Design Patterns: elements of
    reusable object-oriented software} by Gamma, et al. (Typically known of Gang of Four or GoF).
    The book describes 23 patterns.
\end{frame}

%----------- slide --------------------------------------------------%
\begin{frame}
  \frametitle{What is a Design Pattern?}
  
  A design pattern is not code. It's a description of how code (typically a class)
  should function. It describes a broadly useful approach to solving a common problem.
  
  \vspace{5mm}
  Patterns are generally language-independent, although their implementations are not.
  They are typically language paradigm-dependent, so that OO patterns might not apply to 
  other language types. 
  
  \vspace{5mm}
  

  
\end{frame}

%----------- slide --------------------------------------------------%
\begin{frame}
  \frametitle{So what?}
  
  The primary test of the utility of a particular design pattern is that it
  \textbf{works}. It solves the problem it's intended to solve. When you're confronted 
  by a problem that is probably pretty common, look to see if there is a well known pattern
  that solves it.
  
  \vspace{5mm}
  Conversely, it's good to know about common patterns so that you recognise common problems when you 
  see them. Using a well-known pattern also helps other programmers understood what you are doing with your code.
     
\end{frame}

%----------- slide --------------------------------------------------%
\begin{frame}
  \frametitle{Criticisms}
  
  Some criticsims of design patterns have been raised.
  
  \begin{itemize}
    \item Design patterns merely point out shortcomings in particular languages.
    \item Patterns lead to needless complexity in software design.
  \end{itemize}

  \end{frame}
%----------- slide --------------------------------------------------%
\begin{frame}
  \frametitle{The Singleton Pattern} 
  
  One well known pattern is the \emph{Singleton}.
  
  \vspace{5mm}
  ``Ensure a class only has one instance and provide a global point of 
  access to it.'' (Gof)
  
  \vspace{5mm}
  Applications for a singleton include a central logging object, a configuration object,
  or a cache.
  
  \vspace{5mm}
  In Python, \texttt{None} is a singleton.

  \end{frame}

%----------- slide --------------------------------------------------%
\begin{frame}[fragile]
  \frametitle{Singleton: Classic Implementation}

  \begin{verbatim}
  class SoloGoF:
    _instance = None  # This is a class variable

    def __init__(self):
        raise RuntimeError('Use instance() instead.')

    @classmethod
    def instance(cls):
        if not cls._instance:
            cls._instance = cls.__new__(cls)
            # extra initialisation can happen here
        return cls._instance
    
    solo = SoloGoF.instance()    

  \end{verbatim}
 \end{frame} 
 %----------- slide --------------------------------------------------%
\begin{frame}[fragile]
  \frametitle{Singleton: A More Pythonic Implementation}

  \begin{verbatim}
  class SoloPy:
    _instance = None

    def __new__(cls):
        if not cls._instance:
            cls._instance = super(SoloPy, cls).__new__(cls)
            # extra initialisation can happen here
        return cls._instance

    solo3 = SoloPy()
  \end{verbatim}

\end{frame}

%----------- slide --------------------------------------------------%
\begin{frame}
  \frametitle{Programming Activity}
  
  \begin{enumerate}
    \item Pull the course materials repo.
    \item Create a new branch, \texttt{10-practical} in your practicals repo.
    \item Add a subdirectory,  \texttt{10-practical} and copy \texttt{10-practical.ipynb} from the class materials into it.
    \item Open a shell, cd to this directory, and run \texttt{jupyter notebook} to open the notebook. Complete the first questions.
    \item We will discuss results in 20ish minutes.
  \end{enumerate}      
\end{frame}

%----------- slide --------------------------------------------------%
\begin{frame}
  \frametitle{Criticisms of the Singleton}
  If you Google for ``singelton antipattern'', you'll get a lot of results.
  \begin{itemize}
    \item It's too complicated. If you only want one instance, just create one instance.
    \item Singletons are thinly disguised globals.
    \item Singletons are inflexible. What if you really need two?
    \item It's difficult to unit test singletons, since you can't control their instantiation.
    \item It's difficult to test objects that depend on singletons, especially if they contain state.
  \end{itemize}      
\end{frame}

%----------- slide --------------------------------------------------%
\begin{frame}
  \frametitle{Singletons Using Modules}
  In Python, we have \emph{modules} that we can \emph{import}. Recall that a module's
  code is only executed one time and the results are cached for subsequent imports. In effect,
  modules are singletons.
\end{frame} 

%----------- slide --------------------------------------------------%
\begin{frame}[fragile]
  \frametitle{Singleton: Module Implementation}

  \begin{verbatim}
  file: solomod.py
  --------
  class _SoloMod:
    def __init__(self):
        # do some init stuff, idk
        pass
  
  solomod = _SoloMod()  
  
  # alternatively:    
  class SoloMod:
    def __new__(cls):
        return solomod
        
  # in a client module:
  import solomod
  
  solo = solomod.solomod
     
  \end{verbatim}
\end{frame}

\end{document}
