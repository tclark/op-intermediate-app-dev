% Beamer slide template prepared by Tom Clark <tom.clark@op.ac.nz>
% Otago Polytechnic
% Dec 2012

\documentclass[10pt]{beamer}
\usetheme{Dunedin}
\usepackage{graphicx}
\usepackage{fancyvrb}

\newcommand\codeHighlight[1]{\textcolor[rgb]{1,0,0}{\textbf{#1}}}

\title{Testing}

\author[IN608]{Intermediate Application Development}
\institute[Otago Polytechnic]{
  Otago Polytechnic \\
  Dunedin, New Zealand \\
  Kaiako: Tom Clark
}
\date{}
\begin{document}

%----------- titlepage ----------------------------------------------%
\begin{frame}[plain]
  \titlepage
\end{frame}

%----------- slide --------------------------------------------------%
\begin{frame}
  \frametitle{Introduction}
  
  Testing is important, but traditionally programmers aren't good at it. Modern
  programming practice puts more responsibility for testing, however. Thankfully
  (and probably as a result of this), tooling to automate tests is available.
  
  \vspace{5mm}
  Types of tests
  \begin{itemize}
    \item Unit tests
    \item Integration tests
    \item Acceptance tests
  \end{itemize}
  We will focus on unit tests, but much of this can also be applied to 
  integration tests. 
\end{frame}

%----------- slide --------------------------------------------------%
\begin{frame}
  \frametitle{Automated testing}
  
  The only sensible way to handle unit tests is to automate them. This
  
  \begin{itemize}
    \item ensures that tests are performed;
    \item makes tests consistent;
    \item guides development;
    \item provides de facto documentation.
  \end{itemize}
    
  \end{frame}

%----------- slide --------------------------------------------------%
\begin{frame}
  \frametitle{Components of Automated Tests}
  
  
  
  \begin{itemize}
    \item Test cases
    \item Test fixtures
    \item Test suites
    \item Test runners
  \end{itemize}
  
  We will look at Python's \texttt{unittest} module and see how it provides these.
  There are other options, but \texttt{unittest} is provided in the standard library. 
    
  \end{frame}


%----------- slide --------------------------------------------------%
\begin{frame}[fragile]
  \frametitle{Example}
  
  \begin{verbatim}
    
    class Multiplier:
        def __init__(self, factor):
            self.factor = factor
            
        def multiply(self, num):
            return self.factor * num    
    
   \end{verbatim} 
    Let's test this.
\end{frame}
%----------- slide --------------------------------------------------%
\begin{frame}[fragile]
  \frametitle{Example}
  
  
  \begin{verbatim}
    import unittest
    
    class TestMultiplier(unittest.TestCase):
        def setUp(self):
            self.m = Multiplier(2)
                        
        def test_multiply(self):
            result = self.m.multiply(2)
            self.assertIsInstance(result, (int, float, complex))    
            self.assertEqual(result, 4)
   \end{verbatim} 
    
\end{frame}

%----------- slide --------------------------------------------------%
\begin{frame}[fragile]
  \frametitle{Testing Exceptions}
  
  
  \begin{verbatim}
    import unittest
    
    class TestMultiplier(unittest.TestCase):
        ...
                                
        def test_multiply_raises(self):
            with self.assertRaises(TypeError):
                self.m.multiply(2)
 
   \end{verbatim} 
    
\end{frame}
%----------- slide --------------------------------------------------%
\begin{frame}[fragile]
  \frametitle{Skipping Tests}
  
  
  Sometimes you want to skip a test, or you expect a
  test to fail. You can decorate a test with one of these.
  \begin{verbatim}
  
  @unittest.skip('message')
  
  @unittest.skipIf(condition, 'message')
  
  @unittest.expectedFailure
  
  \end{verbatim} 
    
\end{frame}

%----------- slide --------------------------------------------------%
\begin{frame}[fragile]
  \frametitle{Skipping Tests}
  
  
  Sometimes you want to skip a test, or you expect a
  test to fail. You can decorate a test with one of these.
  \begin{verbatim}
  
  @unittest.skip('message')
  
  @unittest.skipIf(condition, 'message')
  
  @unittest.expectedFailure
  
  \end{verbatim} 
    
\end{frame}

%----------- slide --------------------------------------------------%
\begin{frame}[fragile]
  \frametitle{Organising Tests}
  
  
  There is more than one approach to this, but here is 
  a good overall strategy.
  \begin{verbatim}
  
  project_root/
      |
      |- multiplier.py
      | ...
      |- tests/
          |
          |- test_multiplier.py
          | ...
  
  \end{verbatim} 
    
\end{frame}

%----------- slide --------------------------------------------------%
\begin{frame}[fragile]
  \frametitle{Organising Tests}
  
  
  Inside \texttt{test\_multiplier.py}, we have
  \begin{verbatim}
  import unittest
  from multiplier import Multiplier
  
  class TestMultiplier(unittest.TestCase):
      ...
  \end{verbatim} 
    
\end{frame}

%----------- slide --------------------------------------------------%
\begin{frame}[fragile]
  \frametitle{Running Tests}
  
  
  From our project root directory, we can use commands like
  
  \begin{verbatim}
  python -m unittest tests/test_multiplier -v
  python -m unittest discover
  \end{verbatim} 
    
\end{frame}

%----------- slide --------------------------------------------------%
\begin{frame}
  \frametitle{Programming Activity}
  
  \begin{enumerate}
    \item Pull the course materials repo.
    \item Create a new branch, \texttt{18-practical} in your practicals repo.
    \item Copy the subdirectory, \texttt{18-practical} from the class materials into your repo.
    \item See the README for directions.
    \item We will discuss results in 20ish minutes.
  \end{enumerate}      
\end{frame}
  


%----------- slide --------------------------------------------------%
\begin{frame}[fragile]
  \frametitle{Mocking}
  
  Consider this class.
  \begin{verbatim}
  
  class UserManager:
      def get_user_name(self, user_id):
          user  = db.get_user(user_id)
          return user.name
                      
  \end{verbatim} 
  
  This class is hard to test since it relies on an external resource, \texttt{db}.
   
\end{frame}
%----------- slide --------------------------------------------------%
\begin{frame}[fragile]
  \frametitle{Mocking}
  
  \texttt{unittest.mock} helps with this problem.
  
  \begin{verbatim}
  from unittest.mock import Mock
  
  testuser = Mock()
  testuser.name = 'Joe Bloggs'
  db = Mock()
  db.get_user.return_value = testuser
  
  class UserManager:
      def get_user_name(self, user_id):
          user = db.get_user(user_id)
          return user.name
                      
  \end{verbatim} 
  
  \texttt{Mock()} provides all-purpose stand in objects for use in testing and development.
   
\end{frame}

%----------- slide --------------------------------------------------%
\begin{frame}[fragile]
  \frametitle{Mocking in Unit Tests}
  
  We can also use mocks in unit tests
  
  \begin{verbatim}
  from unittest.mock import patch
  import user_manager
  class TestUserManager:
      
      ...
      
      @patch('user_manager.db')
      def test_get_user_name(self, mock_db):
          testuser = Mock()
          testuser.name = 'Joe Bloggs'
          mock_db.get_user.return_value = testuser
          assertEqual(self.usermanager.get_user_name(1),
              'Joe Bloggs')                     
  \end{verbatim} 
  
  The mock objects are used in the test without any modification to the \texttt{UserManager} code.
   
\end{frame}

%----------- slide --------------------------------------------------%
\begin{frame}
  \frametitle{References}
  
  \begin{itemize}
    \item unittest: \url{https://docs.python.org/3/library/unittest.html}
    \item unittest.mock: \url{https://docs.python.org/3/library/unittest.mock.html}
    \item RealPython article about mock: \url{https://realpython.com/python-mock-library/}
  \end{itemize}      
\end{frame}

\end{document}
