% Beamer slide template prepared by Tom Clark <tom.clark@op.ac.nz>
% Otago Polytechnic
% Dec 2012

\documentclass[10pt]{beamer}
\usetheme{Dunedin}
\usepackage{graphicx}
\usepackage{fancyvrb}

\newcommand\codeHighlight[1]{\textcolor[rgb]{1,0,0}{\textbf{#1}}}

\title{Object Serialisation}

\author[IN608]{Intermediate Application Development}
\institute[Otago Polytechnic]{
  Otago Polytechnic \\
  Dunedin, New Zealand \\
  Kaiako: Tom Clark
}
\date{}
\begin{document}

%----------- titlepage ----------------------------------------------%
\begin{frame}[plain]
  \titlepage
\end{frame}

%----------- slide --------------------------------------------------%
\begin{frame}
  \frametitle{Introduction}
  
  Objects are great, but sometimes we need to move them around. Maybe we 
  want to save an object to a file so we can read it in during a later session.
  Perhaps we want to send the object over a network to a remote process. 
  
  \vspace{5mm}
  To do this, we need to convert the object into a \emph{serial} format,
  like a sequence of bytes or a string. We also need a way to convert the object from 
  its serial format into a functioning object instance.
\end{frame}

%----------- slide --------------------------------------------------%
\begin{frame}
  \frametitle{Approaches to Serialisation}
  
  We will look at two ways to serialise Python objects.
  
  \begin{itemize}
    \item Binary format: Python's \texttt{pickle} module
    \item String format: Python's \texttt{json} module and related tooling
  \end{itemize}
    
  \end{frame}

%----------- slide --------------------------------------------------%
\begin{frame}
  \frametitle{Pickle}
  
  Python's standard \texttt{pickle} module saves Python built-in types and
  some custom objects in a binary format.
  
  \begin{itemize}
    \item Pros:
      \begin{itemize}
        \item Easy to use
        \item Efficient
      \end{itemize}  
    \item Cons:
      \begin{itemize}
        \item Python-specific
        \item Can't save or load some values 
      \end{itemize}  
  \end{itemize}
  
  Reference: \url{https://docs.python.org/3/library/pickle.html}
    
  \end{frame}

%----------- slide --------------------------------------------------%
\begin{frame}
  \frametitle{Pickle Versions}
  
  There are currently six versions of the \texttt{pickle} protocol
  
  \begin{itemize}
    \item Versions 0 -2: Too old, don't care
    \item Version 3: introduced in Python 3.0
    \item Version 4: introduced in Python 3.4
    \item Version 5: introduced in Python 3.9
  \end{itemize}
  
  New versions have improved features, but data pickled with a high version
  can't be unpickled in older interpreters that don't support the protocol version.
  You can specify the protocol version used when pickling.
    
  \end{frame}

%----------- slide --------------------------------------------------%
\begin{frame}
  \frametitle{Basic Pickle Functions}
  
  
  
  \begin{itemize}
    \item \texttt{pickle.dump()}: Saves data to a file
    \item \texttt{pickle.dumps()}: Returns a byte string of pickled data
    \item \texttt{pickle.load()}: Loads from a file
    \item \texttt{pickle.loads()}: Loads from a byte string
  \end{itemize}
  
  
    
  \end{frame}


%----------- slide --------------------------------------------------%
\begin{frame}[fragile]
  \frametitle{Example}
  
  \begin{verbatim}
    import pickle
    
    class PickleMe:
        def __init__(self, data):
            self.data = data
    
    example = PickleMe(42)
    pickle.dump(example, open('save.p', 'wb'))
    
    example = None
    example = pickle.load(open('save.p', 'rb'))
    print(example.data) # prints 42       
    
   \end{verbatim} 
    
\end{frame}

%----------- slide --------------------------------------------------%
\begin{frame}
  \frametitle{Security Concerns}
  
  Pickled data is vulnerable to be altered. Don't load untrusted pickled data.
  If you need to send and receive pickled data over an untrusted network use
  something like the \texttt{hmac} module to cryptographically authenticate the data.
    
  \end{frame}
  
%----------- slide --------------------------------------------------%
\begin{frame}
  \frametitle{Programming Activity}
  
  \begin{enumerate}
    \item Pull the course materials repo.
    \item Create a new branch, \texttt{17-practical} in your practicals repo.
    \item Add a subdirectory,  \texttt{17-practical} and copy \texttt{17-practical.ipynb} from the class materials into it.
    \item Open a shell, cd to this directory, and run \texttt{jupyter notebook} to open the notebook. 
    \item We will discuss results in 20ish minutes.
  \end{enumerate}      
\end{frame}
  

%----------- slide --------------------------------------------------%
\begin{frame}
  \frametitle{Serialising with JSON}
  
  Javascript Object Notation (JSON) is a common data serialisation format.
  Python's \texttt{json} module has functions corresponding to those we saw in 
  \texttt{pickle}.
  
  \begin{itemize}
    \item \texttt{json.dump()}: Saves data to a file
    \item \texttt{json.dumps()}: Returns a JSON-formatted string of data
    \item \texttt{json.load()}: Loads from a file
    \item \texttt{json.loads()}: Loads from a JSON string
  \end{itemize}

   Reference: \url{https://docs.python.org/3/library/json.html}

  
\end{frame}
%----------- slide --------------------------------------------------%
 \begin{frame}
  \frametitle{Pros and Cons}
  
  \begin{itemize}
    \item Pros:
      \begin{itemize}
        \item Not Python-dependent
        \item Easy
        \item Extensible
      \end{itemize}  
    \item Cons:
      \begin{itemize}
        \item Can't represent some Python values without additional code
        \item Can't represent custom classes without additional code
      \end{itemize}  
  \end{itemize}
\end{frame}
%----------- slide --------------------------------------------------%
\begin{frame}
  \frametitle{Values that serialise to JSON}
  
  \begin{itemize}
    \item integers
    \item floats
    \item strings
    \item booleans
    \item list
    \item dictionaries\footnote{Sort of. JSON dictionaries have to have string keys.}
    \item \texttt{None}
   \end{itemize}
\end{frame} 

%----------- slide --------------------------------------------------%
\begin{frame}[fragile]
  \frametitle{Example}
  
  \begin{verbatim}
    import json
    
    pets = {'cat': 'Lola',
            'dog': 'Star',
            'fish': 'My fish don't have names.'
           }    
           
    json_pets = json.dumps(pets)
    
    restored_pets = json.loads(json_pets)       
    
   \end{verbatim} 
    
\end{frame}

%----------- slide --------------------------------------------------%
\begin{frame}[fragile]
  \frametitle{Extending JSONEncoder}
  
  If we want to handle a type not supported by default, we 
  can extend the default encoder.
  \begin{verbatim}
    import json
    
    z = 1 + 2j  # complex number
    class ComplexEncoder(json.JSONEncoder):
        def default(self, obj):
            if isinstance(obj, complex):
                return [obj.real, obj.imag] 
            return json.JSONEncoder.default(self, obj)
            
    json_z = json.dumps(z, cls=ComplexEncoder)               
    
    
   \end{verbatim} 
    
\end{frame}

%----------- slide --------------------------------------------------%
\begin{frame}[fragile]
  \frametitle{Adding Serialisation to a Class}
  
  Since the `json` module can't handle custom classes, a typical approach
  is to add serialisation methods to the class.
  \begin{verbatim}
  class Foo:
      def __init__(self, a, b):
          self.a = a
          self.b = b
          
      def dumps(self):
          return json.dumps({'a': self.a, 'b': self.b})
          
      @classmethod
      def loads(cls, js):
          data = json.loads(js)
          return cls(data['a'], data['b'])         
                     
  \end{verbatim} 
   
  We basically just manually make our objects handle their own serialisation. 
\end{frame}

%----------- slide --------------------------------------------------%
\begin{frame}[fragile]
  \frametitle{More Robust Serialisation}
  
  The pip-installable module \texttt{marshmallow} provides 
  a more robust approach to JSON (and other) serialisation.
  \begin{verbatim}
  from marshmallow import Schema, fields
  
  class Foo:
      def __init__(self, a, b):
          self.a = a
          self.b = b
          
  class FooSchema(Schema):
      a = fields.Str(required=True)
      b = fields.Integer()
      
      @post_load
      def make_foo(self, data, **kwargs):
          return Foo(**data)         
                     
  \end{verbatim} 
   
   
\end{frame}

%----------- slide --------------------------------------------------%
\begin{frame}[fragile]
  \frametitle{More Robust Serialisation}
  
  \begin{verbatim}
  
  foo = Foo('hello', 6)
  foo_schema = FooSchema()
  serialised = foo_schema.dump(foo)  # returns a dict, not json
  deserialised = foo_schema.load(serialised)
  
          
                     
  \end{verbatim} 
  
   Reference: \url{https://marshmallow.readthedocs.io/en/stable/index.html}
   
   
\end{frame}


\end{document}
