% Beamer slide template prepared by Tom Clark <tom.clark@op.ac.nz>
% Otago Polytechnic
% Dec 2012

\documentclass[10pt]{beamer}
\usetheme{Dunedin}
\usepackage{graphicx}
\usepackage{fancyvrb}

\newcommand\codeHighlight[1]{\textcolor[rgb]{1,0,0}{\textbf{#1}}}

\title{Async IO and Coroutines}

\author[IN608]{Intermediate Application Development}
\institute[Otago Polytechnic]{
  Otago Polytechnic \\
  Dunedin, New Zealand \\
  Kaiako: Tom Clark
}
\date{}
\begin{document}

%----------- titlepage ----------------------------------------------%
\begin{frame}[plain]
  \titlepage
\end{frame}

%----------- slide --------------------------------------------------%
\begin{frame}
  \frametitle{Introduction}
  
  Last week we looked at threading.
  \begin{itemize}
    \item Multiple concurrent threads of execution.
    \item Threads can speed up an I/O bound program.  
    \item Threads may be swapped in and out at any time. We don't
    control when threads are executed. This can lead to challenging
    bugs. 
    
    \vspace{3mm}
    \emph{But}
    
    \item We can write programs in a familiar way and add threading
    to parts of the program that benefit from it.
  \end{itemize}  
     
\end{frame}

%----------- slide --------------------------------------------------%
\begin{frame}
  \frametitle{A different approach: Async IO}

  Often we write code that blocks, typically while waiting on IO. It generally
  clear to us where this happens in our code. When using an async IO style of 
  coding, we explicitly mark these points in our code so that execution can be
  yielded to some other part of the process. This is called \emph{cooperative multitasking}.

\end{frame}

%----------- slide --------------------------------------------------%
\begin{frame}[fragile]
  \frametitle{Review: Threading}
  
 \begin{verbatim}
  import threading
  from time import sleep
  
  def slow(x):
      sleep(10)
      return x
  
  def do_threaded_tasks(num):
      threads = []
      for n in range(num):
          t = threading.Thread(target=slow, args=(n,))
          threads.append(t)
          t.start() 
      for t in threads:
          t.join()                                                                                                             
 \end{verbatim} 
 One key thing here is that the function \texttt{slow()} doesn't
 know or care that it will be run in a seperate thread.
    
\end{frame}

%----------- slide --------------------------------------------------%
\begin{frame}[fragile]
  \frametitle{An AsyncIO Version}
  
  \begin{verbatim}
   import asyncio

    async def slow(n):
        await asyncio.sleep(3)
        return n

    async def run_async_tasks(n):
        tasks = [asyncio.create_task(slow(i)) for i in range(n)]
        results = await asyncio.gather(*tasks)
        return results

    if __name__ == '__main__':
        results = asyncio.run(run_async_tasks(5))
        print(results)
     
  \end{verbatim}
    
  \end{frame}

%----------- slide --------------------------------------------------%
\begin{frame}
  \frametitle{What's going on here?}

  The main new thing here is the addition of the words \texttt{async} and
  \texttt{await}. These became reserved words in Python as of version 3.7.
  
  \vspace{5mm}
  \texttt{async def} defines a \emph{coroutine}. A typical Python function is
  a \emph{subroutine}. A subroutine runs from start to finish in an uninterrupted 
  manner (threads notwithstanding). A coroutine can suspend and resume execution 
  later from where it left off.
  
  \vspace{5mm}
  We must use \texttt{await} to call a coroutine and only coroutines can call other
  coroutines. The use of \texttt{await} signals the event loop that the calling coroutine
  can be suspended until the result of the \texttt{await}ed call is ready.
  
  \vspace{5mm}
  Also, \texttt{asyncio.run()} provides an event loop in which our coroutines can be run.
\end{frame}

%----------- slide --------------------------------------------------%
\begin{frame}
  \frametitle{}

  On the one hand, this code is simpler that working with threads. Everything
  runs in one thread and we don't have to worry about race conditions or deadlocks.
  
  \vspace{5mm}
  On the other hand, you can't just take some IO bound code and speed it up by
  sprinkling some \texttt{await}s on it. You have to plan for how you're going to 
  use coroutines from the bottom up. 
  \end{frame}

 
%----------- slide --------------------------------------------------%
\begin{frame}[fragile]
  \frametitle{Example, HTTP Requests}
  
  \begin{verbatim}
  import asyncio
  from aiohttp import ClientSession

  async def get_web(session, url):
      resp =  await session.get(url)
      content = await resp.text()
      return content

  async def web_async_tasks():
      urls = ['https://google.com', 'https://github.io']
      async with ClientSession() as session:
          tasks = [asyncio.create_task(get_web(session, url)) 
              for url in urls]
          results = await asyncio.gather(*tasks)
          return results

  if __name__ == '__main__':
      results = asyncio.run(web_async_tasks())
  \end{verbatim}
    
  \end{frame}
  
%----------- slide --------------------------------------------------%
\begin{frame}
  \frametitle{}
  

  This example isn't particularly harder than a non-async IO version. The
  key difference is that we needed a  library, \texttt{aiohttp} made for use
  in async IO code.
  
  \vspace{5mm}
  We could have used the excellent \texttt{requests} library to make our HTTP
  calls. But we can't \texttt{await} a call to \texttt{requests.get()} because
  it's not a coroutine. We can still use \texttt{requests}, but the calls will
  block

  \end{frame}

%----------- slide --------------------------------------------------%
\begin{frame}
  \frametitle{Programming Activity}
  
  \begin{enumerate}
    \item Pull the course materials repo.
    \item Create a new branch, \texttt{23-practical} in your practicals repo.
    \item Copy the subdirectory, \texttt{23-practical} from the class materials into your repo.
    \item See the README for directions.
    \item We will discuss results in 20ish minutes.
  \end{enumerate}      
\end{frame}
  
%----------- slide --------------------------------------------------%
\begin{frame}
  \frametitle{Another Pattern: Producer/Consumer}
  
  A design pattern that works well in an async IO context is 
  \emph{producer/consumer}. It comprises the following parts:
  
  \begin{itemize}
    \item a set of \emph{producers} that prepares data;
    \item a buffer (FIFO) that holds the data produced by the producers;
    \item a set of \emph{consumers} that takes data from the buffer and 
    processes it in some way.
  \end{itemize}  
        
\end{frame}
  
%----------- slide --------------------------------------------------%
\begin{frame}[fragile]
  \frametitle{Producer}
  
  \begin{verbatim}
  async def produce(q):
      num  = randint(0, 5)
      for i in range(num):
          await asyncio.sleep(2)
          await q.put(i)
      
  \end{verbatim}
    
\end{frame}
  
%----------- slide --------------------------------------------------%
\begin{frame}[fragile]
  \frametitle{Consumer}
  
  \begin{verbatim}
  async def consume(q):
      while True:
          await asyncio.sleep(1)
          val = await q.get()
          print(val)
          q.task_done()      
  \end{verbatim}
    
\end{frame}

%----------- slide --------------------------------------------------%
\begin{frame}[fragile]
  \frametitle{Assembling the Pattern}
  
  \begin{verbatim}
   async def main():
       q = asyncio.Queue()
       prods = [asyncio.create_task(produce(q)) 
           for _ in range(N_PRODUCERS)]
       cons = [asyncio.create_task(consume(q)) 
           for _ in range(N_CONSUMERS)]
       await asyncio.gather(*prods)
       await q.join()
       for c in cons:
           c.cancel()   
  \end{verbatim}
    
\end{frame}

%----------- slide --------------------------------------------------%
\begin{frame}
  \frametitle{References}
  
  \begin{itemize}
    \item \texttt{asyncio}: \url(https://docs.python.org/3/library/asyncio.html)
    \item Libraries/frameworks for use with async io: \url{https://github.com/timofurrer/awesome-asyncio} 
   \end{itemize}      
\end{frame}



\end{document}
