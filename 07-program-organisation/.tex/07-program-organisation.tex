% Beamer slide template prepared by Tom Clark <tom.clark@op.ac.nz>
% Otago Polytechnic
% Dec 2012

\documentclass[10pt]{beamer}
\usetheme{Dunedin}
\usepackage{graphicx}
\usepackage{fancyvrb}

\newcommand\codeHighlight[1]{\textcolor[rgb]{1,0,0}{\textbf{#1}}}

\title{Program Organisation: \\ Namespaces, Modules, Imports}

\author[IN608]{Intermediate Application Development}
\institute[Otago Polytechnic]{
  Otago Polytechnic \\
  Dunedin, New Zealand \\
  Kaiako: Tom Clark
}
\date{}
\begin{document}

%----------- titlepage ----------------------------------------------%
\begin{frame}[plain]
  \titlepage
\end{frame}

%----------- slide --------------------------------------------------%
\begin{frame}
  \frametitle{Introduction}
  
  My name is Tom. In many parts of the world that is a very common name.
  So when we talk about ``Tom'', who exactly do we mean?
  
  \begin{itemize}
    \item In this room, I'm probably the only person named Tom.
    \item I appear to be the only Tom Clark at OP, so on campus my ``fully qualified'' name identifies me.
    \item There are a lot of Tom Clarks in NZ, but if we say, ``The Tom Clark at Otago Polytech,'' we're talking about 
    me.
  \end{itemize}
  
  \vspace{5mm}
  In the context of a running program, we have a similar problem.         
\end{frame}

%----------- slide --------------------------------------------------%
\begin{frame}[fragile]
  \frametitle{File: \texttt{example.py}}

  \begin{Verbatim}[commandchars=\\\{\}]
    import otagopolytech
    
    tom = "A really cool guy"
    def enclose():
        tom = "I guess he's ok"
        def local():
            tom = "What a jerk"
            print(tom)
        print(tom)
        return local        
    opinion_of_tom = enclose()
    opinion_of_tom()   
    print(tom)
  \end{Verbatim}
  What gets printed?
\end{frame}

%----------- slide --------------------------------------------------%
\begin{frame}
  \frametitle{Namespaces}
  
  In our example, the name \texttt{tom} gets used over and over, but they don't 
  conflict because each one exists in a distinct \emph{namespace}. In a running 
  Python program several namespaces may exist at any one time.
  
  \vspace{5mm}
  When we use a name like \texttt{tom}, Python applies a set of rules for searching 
  namespaces for that name.
  
     
\end{frame}

%----------- slide --------------------------------------------------%
\begin{frame}[fragile]
  \frametitle{Global Namespace}

  \begin{Verbatim}[commandchars=\\\{\}]
    import otagopolytech
    
    \codeHighlight{tom = "A really cool guy"}
    def enclose():
    
    ...
  \end{Verbatim}
  The first occurrence of \texttt{tom} is in a \emph{global} namespace. This
  name is meaningful anywhere in the file. The
  imported module \texttt{otagopolytechnic} actually defines a seperate, distinct
  global namespace.
  
  
\end{frame}
%----------- slide --------------------------------------------------%
\begin{frame}[fragile]
  \frametitle{Local Namespace}

  \begin{Verbatim}[commandchars=\\\{\}]
        ...
        def local():
            \codeHighlight{tom = "What a jerk"}
            print(tom)
        ...    
  \end{Verbatim}
  The innermost occurrence of \texttt{tom} is in a \emph{local} namespace. This
  version of \texttt{tom} is only meaningful in the context of executing this function.  
  
\end{frame}

%----------- slide --------------------------------------------------%
\begin{frame}[fragile]
  \frametitle{Enclosing Namespace}

  \begin{Verbatim}[commandchars=\\\{\}]
        ...
   def enclose():
        \codeHighlight{tom = "I guess he's ok"}    
        def local():
            tom = "What a jerk"
            print(tom)
        print(tom)
        return local  
        ...    
  \end{Verbatim}
  The function \texttt{enclose()} defines an \emph{enclosing} namespace. This
  version of \texttt{tom} is meaningful in the context of the function \texttt{enclose()}
  \textbf{and} within the enclosed function \texttt{local()}.  
\end{frame}

%----------- slide --------------------------------------------------%
\begin{frame}[fragile]
  \frametitle{Builtin Namespace}

  \begin{Verbatim}[commandchars=\\\{\}]
    import otagopolytech
    
    tom = "A really cool guy"
    def enclose():
        tom = "I guess he's ok"
        def local():
            tom = "What a jerk"
            \codeHighlight{print}(tom)
       \codeHighlight{print}(tom)
        return local        
    opinion_of_tom = enclose()
    opinion_of_tom()   
    \codeHighlight{print}(tom)
  \end{Verbatim}
  We didn't do anything to define \texttt{print()}. It is in the \emph{builtin} namespace.
\end{frame}

%----------- slide --------------------------------------------------%
\begin{frame}
  \frametitle{LEGB}

  When the Python interpreter looks for a name like \texttt{tom}, it searches
  the namespaces in the order
  
  \begin{enumerate}
    \item Local
    \item Enclosing
    \item Global
    \item Builtin
  \end{enumerate}  
  
\end{frame}




\begin{frame}
  \frametitle{Programming Activity}
  
  \begin{enumerate}
    \item Pull the course materials repo.
    \item Create a new branch, \texttt{06-practical} in your practicals repo.
    \item Add a subdirectory,  \texttt{06-practical} and copy \texttt{06-practical.ipynb} from the class materials into it.
    \item Open a shell, cd to this directory, and run \texttt{jupyter notebook} to open the notebook. Complete the first questions.
    \item We will discuss results in 30ish minutes.
  \end{enumerate}      
\end{frame}

%----------- slide --------------------------------------------------%
\begin{frame}[fragile]
  \frametitle{Imports}
  
  Suppose I have two files
  \begin{verbatim}
  mod.py
  ------------
  num = 42
  def foo():
      return 'bar'
  
  main.py
  ------------
  import mod
  
  print(mod.num)
  baz = mod.foo()
  \end{verbatim}
  \texttt{mod.py} defines a \emph{module}. The \texttt{import} brings the 
  name \texttt{mod} into \texttt{main.py}'s global namespace and we can access
  its attributes there.
\end{frame}
  
  %----------- slide --------------------------------------------------%
\begin{frame}[fragile]
  \frametitle{Imports}
  
  We can also do this
  \begin{verbatim}
  mod.py
  ------------
  num = 42
  def foo():
      return 'bar'
  
  main.py
  ------------
  from mod import num
  
  print(num)
  \end{verbatim}
  
  In this case we just bring the name \texttt{num} into \texttt{main.py}'s namespace.

\end{frame}

  %----------- slide --------------------------------------------------%
\begin{frame}[fragile]
  \frametitle{Where does import find modules/packages?}
  
  When we use
  \begin{verbatim}
  import mod
  \end{verbatim}
  
  The interpreter needs to find the module or package named \texttt{mod}. It searches
  the following locations in order.
  \begin{enumerate}
    \item \texttt{sys.modules} - a cache of loaded modules.
    \item The current working directory from which the program was invoked.
    \item Any directories listed in the \texttt{PYTHONPATH} environment variable.
    \item A list of installation-dependent directories. You can see these by inspecting \texttt{sys.path}.
  \end{enumerate}
  N.B.: These lists of locations can be modified at runtime, which is sometimes useful but also a security vulnerability if
  you are running untrusted code.  

\end{frame}

 %----------- slide --------------------------------------------------%
\begin{frame}[fragile]
  \frametitle{Modules}
  
  As we've seen before, a Python \emph{module} is just a file with Python code in it.
  It has it's own global namespace and anything defined at the global level may be imported.
  
  \begin{verbatim}
  mod.py
  ------------
  num = 42
  def foo():
      msg = 'bar'
      return msg
  
  \end{verbatim}
  
  So in this case we can \texttt{import mod} and get access to \texttt{mod.num} and \texttt{mod.foo()}.
  We can't import \texttt{foo()}'s local variable, \texttt{msg}.

\end{frame}

 %----------- slide --------------------------------------------------%
\begin{frame}[fragile]
  \frametitle{Packages}
  
  We can bundle multiple modules together in a directory, and we call this a \emph{package}
  
  \begin{verbatim}
     main.py
     mypackage/
       |
       | mod1.py
       | mod2.py
  \end{verbatim}
  
  Then, in \texttt{main.py} we can use
  
  \begin{verbatim}
  import mypackage.mod1
  import mypackage.mod2
  \end{verbatim}
\end{frame}

 %----------- slide --------------------------------------------------%
\begin{frame}[fragile]
  \frametitle{Packages}
  
  If \texttt{mypackage/} contains a file called \texttt{\_\_init\_\_.py}, anything
  in that file can be imported with just \texttt{import mypackage}.
  
  \begin{verbatim}
     main.py
     mypackage/
       | __init__.py
       | mod1.py
       | mod2.py
  \end{verbatim}
  
  In \texttt{main.py}:
  
  \begin{verbatim}
  import mypackage
  import mypackage.mod1
  import mypackage.mod2
  \end{verbatim}
  
  Notice that we still have to explicitly import \texttt{mypackage.mod1} and \texttt{mypackage.mod2}.
\end{frame}

 %----------- slide --------------------------------------------------%
\begin{frame}
  \frametitle{Further Reading}
  
  \begin{itemize}
    \item \url{https://realpython.com/python-namespaces-scope/}
    \item \url{https://realpython.com/python-import/}
    \item \url{https://realpython.com/python-modules-packages/}
  \end{itemize}  
  
\end{frame}
\end{document}
