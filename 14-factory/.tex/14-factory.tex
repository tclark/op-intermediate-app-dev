% Beamer slide template prepared by Tom Clark <tom.clark@op.ac.nz>
% Otago Polytechnic
% Dec 2012

\documentclass[10pt]{beamer}
\usetheme{Dunedin}
\usepackage{graphicx}
\usepackage{fancyvrb}

\newcommand\codeHighlight[1]{\textcolor[rgb]{1,0,0}{\textbf{#1}}}

\title{The Factory Method Pattern \\ and \\ Abstract Factory Pattern}

\author[IN608]{Intermediate Application Development}
\institute[Otago Polytechnic]{
  Otago Polytechnic \\
  Dunedin, New Zealand \\
  Kaiako: Tom Clark
}
\date{}
\begin{document}

%----------- titlepage ----------------------------------------------%
\begin{frame}[plain]
  \titlepage
\end{frame}

%----------- slide --------------------------------------------------%
\begin{frame}[fragile]
  \frametitle{Introduction: Factory Method}
  
  I always used to hate this pattern. Most descriptions I've seen for it are
  confusing. The UML diagram is confusing. It always seemed to me like it was 
  making things \emph{more} complicated.
  
  \vspace{5mm}
  ``Define an interface for creating an object, but let subclasses decide which class
  to instantiate. Factory Method lets a class defer instantiation to subclasses.'' \\
  (\emph{GoF})
  
  That didn't really help. What we need is a good example.

  
  \end{frame}
 %----------- slide --------------------------------------------------%
\begin{frame}
  \frametitle{Our Problem}
  
  I work on games that are available around the world. Naturally, we
  want to be able to present those games to players in the preferred 
  languages. We don't know what language we're going to use until 
  runtime. Clearly we need some sort of translator class.  
  \end{frame}
  

%----------- slide --------------------------------------------------%
\begin{frame}[fragile]
  \frametitle{Let's Do it Wrong}

  \begin{verbatim}
  class Translator:
      def localise(self, lang, expression):
          if lang == 'en':
              # load up English language stuff
              # look up expression
              # return English version
          elif lang == 'es':
              # load up Spanish language stuff
              # look up expression
              # return Spanish version
          ...
       
  \end{verbatim}
  You see where this is going. What's wrong with this?
 \end{frame} 

%----------- slide --------------------------------------------------%
\begin{frame}
  \frametitle{What's Wrong?}

  \begin{itemize}
    \item What's the process for adding a language? (Open-closed)
    \item What if we change the ``\texttt{load up X language stuff}'' process? (Single responsibility)
  \end{itemize}
 \end{frame} 


%----------- slide --------------------------------------------------%
\begin{frame}[fragile]
  \frametitle{Improved Version}

  
  \begin{Verbatim}[commandchars=\\\{\}]
  class Translator:
      def localise(self, lang, expression):
          translator = \codeHighlight{self.get_translator(lang)}
          return translator.interpret(expression)
          
  # using the Translator
  weather = 'WEATHER.CLOUDY'
  localiser = Translator()
  print(localiser.localise(player.preferred_language, weather))        

  \end{Verbatim}
 \end{frame} 

%----------- slide --------------------------------------------------%
\begin{frame}[fragile]
  \frametitle{Get Translator Method}

  How do we implement \texttt{get\_translator()}?

  \begin{verbatim}
  
      def get_translator(self, lang):
          if lang == 'en':
              return English()
          elif lang == 'es':
              return Spanish() 
          ...                  
  \end{verbatim}
  Not bad, but it could be better.
 \end{frame} 
 
%----------- slide --------------------------------------------------%
\begin{frame}
  \frametitle{A better process}
  
  \begin{itemize}
    \item Set up directories. like \texttt{localisations/en} and \texttt{localisations/es}.
    \item Each directory contains files, like \texttt{weather}, with tables to look up expressions and their localised values.
    \item Now \texttt{get\_translator()} looks for a directory and loads up its files.
    \item Also, switching from one strategy to the other is just a matter of changing the \texttt{get\_translator()} method.
  \end{itemize}      
\end{frame}

%----------- slide --------------------------------------------------%
\begin{frame}
  \frametitle{Taking it further}
  
  Suppose my employer gets bought out by Disney. Our new corporate overlords
  insist that every game include a Jar Jar Binks option where characters speak
  with Jar Jar's accent thing in whatever language.
  
  \vspace{5mm}
  Suppose also that we come up with an algorithm to take test in an arbitrary
  language and transform it into the Jar Jar Binks accent.    
\end{frame}

%----------- slide --------------------------------------------------%
\begin{frame}[fragile]
  \frametitle{Subclassing Translator}

  
  \begin{verbatim}
  class JarJarTranslator(Translator):
      def get_translator(self, lang):
          # This would load up the language info and
          # somehow JarJar-ise it? I never said it was
          # a good idea.
          
  \end{verbatim}
  Notice that we don't have to override the \texttt{localise()} method. We
  just change the factory method \texttt{get\_translator()}.
 \end{frame}  
 
%----------- slide --------------------------------------------------%
\begin{frame}
  \frametitle{Programming Activity}
  
  \begin{enumerate}
    \item Pull the course materials repo.
    \item Create a new branch, \texttt{14-practical} in your practicals repo.
    \item Add a subdirectory,  \texttt{14-practical} and copy \texttt{14-practical.ipynb} from the class materials into it.
    \item Open a shell, cd to this directory, and run \texttt{jupyter notebook} to open the notebook. Complete the first two questions.
    \item We will discuss results in 20ish minutes.
  \end{enumerate}      
\end{frame}
%----------- slide --------------------------------------------------%
\begin{frame}[fragile]
  \frametitle{Introduction: Abstract Factory}
  
  The Abstract Factory may be thought of as related to the Factory Method and 
  is sort of is, but it's not a very strong relationship. They are both \emph{Creational
  Patterns}, but that may be as far as it goes.
    
  \vspace{5mm}
  ``Provide an interface for creating \textbf{families} of related or dependent objects without 
  specifying their concrete classes'' \\
  (\emph{GoF})
  
  \vspace{5mm}
  The bit about families of objects is the key. With the Factory Method we wanted one object and we weren't 
  too particular about its concrete type. With the Abstract Factory we want families of related objects.

  
  \end{frame}
  
%----------- slide --------------------------------------------------%
\begin{frame}[fragile]
  \frametitle{Some Abstract Classes}

  
  \begin{verbatim}
  from abc import ABC, abstractmethod
  
  class PetFactory(ABC):
      
      @abstractmethod
      def get_animal(self):
          pass
        
      @abstractmethod  
      def get_pet_food(self):
          pass
      
      @abstractmethod 
      def get_pet_toy(self):
          pass        
  
  \end{verbatim}
\end{frame}  

%----------- slide --------------------------------------------------%
\begin{frame}[fragile]
  \frametitle{More Classes}

  
  \begin{verbatim}
  
  class Animal(ABC):
      pass
      
  class Dog(Animal):
      pass    
      
  class PetFood(ABC):
      pass
      
  class DogRoll(PetFood):
      pass    
      
  class PetToy(ABC):
      pass
      
  class Ball(PetToy):
      pass           
      
  \end{verbatim}
\end{frame}  
  
  
%----------- slide --------------------------------------------------%
\begin{frame}[fragile]
  \frametitle{And Now a Concrete Factory}

  
  \begin{verbatim}
  
  class DogFactory(PetFactory):
  
      def get_animal(self):
          return Dog()
          
      def get_pet_food(self):
          return DogRoll()
          
      def get_pet_toy(self):
          return Ball()          

  
  \end{verbatim}
\end{frame} 
  
%----------- slide --------------------------------------------------%
\begin{frame}[fragile]
  \frametitle{Using the Factory}

  
  \begin{verbatim}  
  def give_a_pet_to(person, factory):
      pet = {'animal': factory.get_animal(),
             'food': factory.get_pet_food(),
             'toy': factory.get_pet_toy()
            }
      person.add_pet(pet)
  
  factory = DogFactory()
  # I have a dog named Star
  give_a_pet_to(tom, factory)         
  
  \end{verbatim}
  
  I also have a cat named Lola. Can you see how we would implement cats?
\end{frame} 

%----------- slide --------------------------------------------------%
\begin{frame}[fragile]
  \frametitle{Some Closing Notes}
  
  In Python and similar dynamic languages we might not bother with the abstract classes. They
  make the design a just little more clear.
  
  \vspace{5mm}
  Similarly, both of these patterns are seen less often in a language like Python than in some other 
  languages, in particular because classes and functions are first-class values and this removes some 
  of the need for these patterns.
  
  \vspace{5mm}
  Finally, a technique called \emph{dependency injection} can serve some of the same purposes, especially in the case of the 
  Factory Method. We won't cover dependency injection in this paper, however.

  
  \end{frame}
  

\end{document}
