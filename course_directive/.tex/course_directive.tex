\documentclass{article}
\usepackage{graphicx}
\usepackage{wrapfig}
\usepackage{inconsolata}
\usepackage{enumerate}
\usepackage{hyperref}
\usepackage[margin = 2.25cm]{geometry}




\begin{document}

\begin{figure}
\includegraphics[width=30mm]{oplogo.png}
\end{figure}

\title{Course Directive\\IN608: Intermediate Application Development Concepts \\Semester One, 2022}
\date{}
\maketitle

\section*{Description}
In this paper we will explore some more advanced programming concepts, such as data structures, concurrency and 
design patterns.  We will also learn some important practical programming methods, including testing and networked
application programming



\section*{Course Information}
\begin{itemize}
  \item 15 Credits
  \item Class sessions: Mondays, 8:00 AM \& Wednesdays, 3:00 PM
  \item D202/D207
\end{itemize}

\section*{Lecturer}
\begin{tabular}{lr}

  % after \\: \hline or \cline{col1-col2} \cline{col3-col4} ...
  Tom Clark &    \\
     Email: & \texttt{tclark@op.ac.nz} \\
        
\end{tabular}

\section*{Course Dates}
\begin{tabular}{ll}
Term 1 (8 weeks) & 21 February - 15 April\\
Term 2 (8 weeks) &  2 May - 24 June\\
\end{tabular}

\newpage 

\section*{Learning Outcomes}
On successful completion of this paper you will be able to:
\begin{enumerate}
  \item Demonstrate sound programming by following design patterns and best practices;
  \item Design and implement full-stack applications using industry relevant programming languages.
\end{enumerate}

\section*{Resources}
\begin{itemize}
	\item Course notes, lecture slides, and lab documents are availble in a GitHub repository published at \\ \url{https://github.com/tclark/op-intermediate-app-dev}.
	\item You will need a GitHub account to submit your assessments. Assessments are available at the GitHub Classroom repositories below.
	    \begin{enumerate}
	    \item: Practicals: \url{https://classroom.github.com/a/R3sKZnEX}
	    \item: Assignment One: \url{https://classroom.github.com/a/Eym227hH}
	    \item: Assignment Two: \url{https://classroom.github.com/a/QYkKMBMZ}
	  \end{enumerate}
	  The submission process will be covered in class.
	\item Programming for this paper will be done in the Python programming language. Setup of your Python environment will be discussed in class. 
	\item There is no text, but assigned readings may be specified by the lecturer.     
\end{itemize}

\pagebreak

\section*{Course Content and Schedule}
This schedule is subject to change.

\renewcommand{\arraystretch}{1.5}
\begin{tabular}{|l|c|l|l|}
\hline
 Week     & Week Start & Session 1 - Monday                 & Session 2 - Wednesday   \\ \hline
 1        & 21 Feb     & Introduction                       & OOP Review              \\ \hline
 2        & 28 Feb     & Data Types                         & Data Structures         \\ \hline
 3        &  7 Mar     & Operator Overloading               & Functional Programming  \\ \hline
 4        & 14 Mar     & Exceptions                         & Modules, Imports        \\ \hline
 5        & 21 Mar     & Otago Anniversary Day              & SOLID                   \\ \hline
 6        & 28 Mar     & Patterns, Singletons               & Decorators              \\ \hline
 7        &  4 Apr     & Iterators                          & Sequences               \\ \hline
 8        & 11 Apr     & Observer                           & Factory                 \\ \hline
 9        & 18 Apr     & Holiday                            & Holiday                 \\ \hline
 10       & 25 Apr     & Holiday                            & Holiday                 \\ \hline
 11       &  2 May     & Project Work                       & Project Work            \\ \hline
 H1       &  9 May     & Serialisation                      & Testing                 \\ \hline
 H2       & 16 May     & Testing                            & Network Sockets         \\ \hline
 12       & 23 May     & Network Sockets                    & Databases, ORM          \\ \hline
 13       & 30 May     & Threading, Forking                 & Async IO                \\ \hline
 14       &  6 Jun     & Queen's Birthday                   & Packaging               \\ \hline
 15       & 13 Jun     & Project Work                       & Project Work            \\ \hline
 16       & 20 Jun     & Project Work                       & Project Work            \\ \hline
\end{tabular}



\section*{Assessment}
There are three assessments in this paper, weighted as follows:


\begin{tabular}{|l|c|c|}
\hline
Assessment                    & Due Date       & Weighting \\ \hline
Practicals                    & 24 Jun         & 20\% \\ \hline
Project 1: Problem  Set       &  6 May         & 30\% \\ \hline
Project 2: Chat Server        & 22 Jun         & 50\%. \\ \hline
\end{tabular}

\section*{Learning Hours}
This course requires 150 hours of learning. This time includes 64 hours of timetabled class time, and 86 
hours of self-directed reading, preparation and completion of assessment work.


\section*{Criteria for Passing}
You must earn an overall average mark of 50\% or better to pass this paper. There must be a genuine 
attempt at all assessments. There are no resits.

\section*{Course Requirements and Expectations}
\subsection*{Attendance}
\begin{itemize}
 \item Students are expected to attend all classes, both lectures and labs, but since we are all dealing with COVID we will be flexible on this.
 \item If you miss a class you should get notes from another student.
 \item If you cannot attend for two or more consecutive sessions, contact the lecturer.
\end{itemize}

\subsection*{Proprietary software}
This class can be completed using only free/open source software (FOSS). Proprietary software is present on lab computers. 


\subsection*{Communication}
Important announcements and discussions about the course, assessments, and scheduling may take place during class sessions.  It is your responsibility to be informed about them.  If you cannot attend a class session, be sure to check with another student.

Your Microsoft Teams and student email is another official communication channel. It is your responsibility to regularly check your student email for important course related material, including changes to class scheduling or assessment details. Not checking will not be accepted as an excuse.



\subsection*{Polytechnic Closure}
In the event that the Polytechnic is closed or has a delayed opening because of snow or bad weather you should not attempt to attend class if it is unsafe to do so. It is possible that your instructor will not be able to attend either, so classes may not physically meet. However, this does not become a holiday. Rather, material will be available on GutHub covering the classes affected by the closure. You are responsible for any material presented in this manner. Information about closure will be posted on the Otago Polytechnic Facebook page \url{https://www.facebook.com/OtagoPoly}.

\subsection*{Group Work and Originality}
Students in the Bachelor of Information Technology degree are expected to hand in original work.  
Students are encouraged to discuss assignments with their fellow students.  However, all assignments 
are to be completed as individual works unless group work is explicitly involved.
Failure to submit your own unique work will be treated as plagiarism.

\subsection*{Referencing}
Appropriate referencing is required for all work.  Referencing standards will be specified by your instructor.

\subsection*{Plagiarism}
Plagiarism is submitting someone else's work as your own.  Plagiarism offences are taken seriously and an
assessment that has been plagiarised may be awarded a zero mark.  A definition of plagiarism is in the Student Handbook,
available online or at the school office.

\subsection*{Submission Requirements}
All assignments are to be submitted by the time, date, and method given when the assignment is issued.

\subsection*{Extensions}
If circumstances are likely to prevent you finishing an assessment on time, contact the lecturer as soon 
as possible, but definitely before the due date.  These must be applied for, and approved, prior to the submission deadline.

\subsection*{Impairment}
In case of sickness contact your lecturer or year co-ordinator as soon as possible, preferably before the test or
assignment is due.  The policy regarding the granting of a mark that considers impaired performance requires a medical
certificate and a medical practitioners signature on a form. You may should refer to the guide on impaired performance
on the student handbook.

\subsection*{Appeals}
If you are concerned about any aspect of your assessment, please approach the lecturer in the first instance.  We support
an open door policy and aim to resolve issues promptly.  Further support is available from the Programme
Manager and Head of School. Otago Polytechnic has a formal process for academic appeals if necessary.

\subsection*{Other Documents}
Regulatory documents relating this course can be found on the Polytechnic website.

\subsection*{Special Resources and Requirements}
If you have any special needs, whether they relate to the course material, the exercises, the assessment, or anything in the course -
then please let your instructor know as soon as possible.

\end{document}
